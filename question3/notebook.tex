
% Default to the notebook output style

    


% Inherit from the specified cell style.




    
\documentclass[11pt]{article}

    
    
    \usepackage[T1]{fontenc}
    % Nicer default font (+ math font) than Computer Modern for most use cases
    \usepackage{mathpazo}

    % Basic figure setup, for now with no caption control since it's done
    % automatically by Pandoc (which extracts ![](path) syntax from Markdown).
    \usepackage{graphicx}
    % We will generate all images so they have a width \maxwidth. This means
    % that they will get their normal width if they fit onto the page, but
    % are scaled down if they would overflow the margins.
    \makeatletter
    \def\maxwidth{\ifdim\Gin@nat@width>\linewidth\linewidth
    \else\Gin@nat@width\fi}
    \makeatother
    \let\Oldincludegraphics\includegraphics
    % Set max figure width to be 80% of text width, for now hardcoded.
    \renewcommand{\includegraphics}[1]{\Oldincludegraphics[width=.8\maxwidth]{#1}}
    % Ensure that by default, figures have no caption (until we provide a
    % proper Figure object with a Caption API and a way to capture that
    % in the conversion process - todo).
    \usepackage{caption}
    \DeclareCaptionLabelFormat{nolabel}{}
    \captionsetup{labelformat=nolabel}

    \usepackage{adjustbox} % Used to constrain images to a maximum size 
    \usepackage{xcolor} % Allow colors to be defined
    \usepackage{enumerate} % Needed for markdown enumerations to work
    \usepackage{geometry} % Used to adjust the document margins
    \usepackage{amsmath} % Equations
    \usepackage{amssymb} % Equations
    \usepackage{textcomp} % defines textquotesingle
    % Hack from http://tex.stackexchange.com/a/47451/13684:
    \AtBeginDocument{%
        \def\PYZsq{\textquotesingle}% Upright quotes in Pygmentized code
    }
    \usepackage{upquote} % Upright quotes for verbatim code
    \usepackage{eurosym} % defines \euro
    \usepackage[mathletters]{ucs} % Extended unicode (utf-8) support
    \usepackage[utf8x]{inputenc} % Allow utf-8 characters in the tex document
    \usepackage{fancyvrb} % verbatim replacement that allows latex
    \usepackage{grffile} % extends the file name processing of package graphics 
                         % to support a larger range 
    % The hyperref package gives us a pdf with properly built
    % internal navigation ('pdf bookmarks' for the table of contents,
    % internal cross-reference links, web links for URLs, etc.)
    \usepackage{hyperref}
    \usepackage{longtable} % longtable support required by pandoc >1.10
    \usepackage{booktabs}  % table support for pandoc > 1.12.2
    \usepackage[inline]{enumitem} % IRkernel/repr support (it uses the enumerate* environment)
    \usepackage[normalem]{ulem} % ulem is needed to support strikethroughs (\sout)
                                % normalem makes italics be italics, not underlines
    

    
    
    % Colors for the hyperref package
    \definecolor{urlcolor}{rgb}{0,.145,.698}
    \definecolor{linkcolor}{rgb}{.71,0.21,0.01}
    \definecolor{citecolor}{rgb}{.12,.54,.11}

    % ANSI colors
    \definecolor{ansi-black}{HTML}{3E424D}
    \definecolor{ansi-black-intense}{HTML}{282C36}
    \definecolor{ansi-red}{HTML}{E75C58}
    \definecolor{ansi-red-intense}{HTML}{B22B31}
    \definecolor{ansi-green}{HTML}{00A250}
    \definecolor{ansi-green-intense}{HTML}{007427}
    \definecolor{ansi-yellow}{HTML}{DDB62B}
    \definecolor{ansi-yellow-intense}{HTML}{B27D12}
    \definecolor{ansi-blue}{HTML}{208FFB}
    \definecolor{ansi-blue-intense}{HTML}{0065CA}
    \definecolor{ansi-magenta}{HTML}{D160C4}
    \definecolor{ansi-magenta-intense}{HTML}{A03196}
    \definecolor{ansi-cyan}{HTML}{60C6C8}
    \definecolor{ansi-cyan-intense}{HTML}{258F8F}
    \definecolor{ansi-white}{HTML}{C5C1B4}
    \definecolor{ansi-white-intense}{HTML}{A1A6B2}

    % commands and environments needed by pandoc snippets
    % extracted from the output of `pandoc -s`
    \providecommand{\tightlist}{%
      \setlength{\itemsep}{0pt}\setlength{\parskip}{0pt}}
    \DefineVerbatimEnvironment{Highlighting}{Verbatim}{commandchars=\\\{\}}
    % Add ',fontsize=\small' for more characters per line
    \newenvironment{Shaded}{}{}
    \newcommand{\KeywordTok}[1]{\textcolor[rgb]{0.00,0.44,0.13}{\textbf{{#1}}}}
    \newcommand{\DataTypeTok}[1]{\textcolor[rgb]{0.56,0.13,0.00}{{#1}}}
    \newcommand{\DecValTok}[1]{\textcolor[rgb]{0.25,0.63,0.44}{{#1}}}
    \newcommand{\BaseNTok}[1]{\textcolor[rgb]{0.25,0.63,0.44}{{#1}}}
    \newcommand{\FloatTok}[1]{\textcolor[rgb]{0.25,0.63,0.44}{{#1}}}
    \newcommand{\CharTok}[1]{\textcolor[rgb]{0.25,0.44,0.63}{{#1}}}
    \newcommand{\StringTok}[1]{\textcolor[rgb]{0.25,0.44,0.63}{{#1}}}
    \newcommand{\CommentTok}[1]{\textcolor[rgb]{0.38,0.63,0.69}{\textit{{#1}}}}
    \newcommand{\OtherTok}[1]{\textcolor[rgb]{0.00,0.44,0.13}{{#1}}}
    \newcommand{\AlertTok}[1]{\textcolor[rgb]{1.00,0.00,0.00}{\textbf{{#1}}}}
    \newcommand{\FunctionTok}[1]{\textcolor[rgb]{0.02,0.16,0.49}{{#1}}}
    \newcommand{\RegionMarkerTok}[1]{{#1}}
    \newcommand{\ErrorTok}[1]{\textcolor[rgb]{1.00,0.00,0.00}{\textbf{{#1}}}}
    \newcommand{\NormalTok}[1]{{#1}}
    
    % Additional commands for more recent versions of Pandoc
    \newcommand{\ConstantTok}[1]{\textcolor[rgb]{0.53,0.00,0.00}{{#1}}}
    \newcommand{\SpecialCharTok}[1]{\textcolor[rgb]{0.25,0.44,0.63}{{#1}}}
    \newcommand{\VerbatimStringTok}[1]{\textcolor[rgb]{0.25,0.44,0.63}{{#1}}}
    \newcommand{\SpecialStringTok}[1]{\textcolor[rgb]{0.73,0.40,0.53}{{#1}}}
    \newcommand{\ImportTok}[1]{{#1}}
    \newcommand{\DocumentationTok}[1]{\textcolor[rgb]{0.73,0.13,0.13}{\textit{{#1}}}}
    \newcommand{\AnnotationTok}[1]{\textcolor[rgb]{0.38,0.63,0.69}{\textbf{\textit{{#1}}}}}
    \newcommand{\CommentVarTok}[1]{\textcolor[rgb]{0.38,0.63,0.69}{\textbf{\textit{{#1}}}}}
    \newcommand{\VariableTok}[1]{\textcolor[rgb]{0.10,0.09,0.49}{{#1}}}
    \newcommand{\ControlFlowTok}[1]{\textcolor[rgb]{0.00,0.44,0.13}{\textbf{{#1}}}}
    \newcommand{\OperatorTok}[1]{\textcolor[rgb]{0.40,0.40,0.40}{{#1}}}
    \newcommand{\BuiltInTok}[1]{{#1}}
    \newcommand{\ExtensionTok}[1]{{#1}}
    \newcommand{\PreprocessorTok}[1]{\textcolor[rgb]{0.74,0.48,0.00}{{#1}}}
    \newcommand{\AttributeTok}[1]{\textcolor[rgb]{0.49,0.56,0.16}{{#1}}}
    \newcommand{\InformationTok}[1]{\textcolor[rgb]{0.38,0.63,0.69}{\textbf{\textit{{#1}}}}}
    \newcommand{\WarningTok}[1]{\textcolor[rgb]{0.38,0.63,0.69}{\textbf{\textit{{#1}}}}}
    
    
    % Define a nice break command that doesn't care if a line doesn't already
    % exist.
    \def\br{\hspace*{\fill} \\* }
    % Math Jax compatability definitions
    \def\gt{>}
    \def\lt{<}
    % Document parameters
    \title{question3}
    
    
    

    % Pygments definitions
    
\makeatletter
\def\PY@reset{\let\PY@it=\relax \let\PY@bf=\relax%
    \let\PY@ul=\relax \let\PY@tc=\relax%
    \let\PY@bc=\relax \let\PY@ff=\relax}
\def\PY@tok#1{\csname PY@tok@#1\endcsname}
\def\PY@toks#1+{\ifx\relax#1\empty\else%
    \PY@tok{#1}\expandafter\PY@toks\fi}
\def\PY@do#1{\PY@bc{\PY@tc{\PY@ul{%
    \PY@it{\PY@bf{\PY@ff{#1}}}}}}}
\def\PY#1#2{\PY@reset\PY@toks#1+\relax+\PY@do{#2}}

\expandafter\def\csname PY@tok@w\endcsname{\def\PY@tc##1{\textcolor[rgb]{0.73,0.73,0.73}{##1}}}
\expandafter\def\csname PY@tok@c\endcsname{\let\PY@it=\textit\def\PY@tc##1{\textcolor[rgb]{0.25,0.50,0.50}{##1}}}
\expandafter\def\csname PY@tok@cp\endcsname{\def\PY@tc##1{\textcolor[rgb]{0.74,0.48,0.00}{##1}}}
\expandafter\def\csname PY@tok@k\endcsname{\let\PY@bf=\textbf\def\PY@tc##1{\textcolor[rgb]{0.00,0.50,0.00}{##1}}}
\expandafter\def\csname PY@tok@kp\endcsname{\def\PY@tc##1{\textcolor[rgb]{0.00,0.50,0.00}{##1}}}
\expandafter\def\csname PY@tok@kt\endcsname{\def\PY@tc##1{\textcolor[rgb]{0.69,0.00,0.25}{##1}}}
\expandafter\def\csname PY@tok@o\endcsname{\def\PY@tc##1{\textcolor[rgb]{0.40,0.40,0.40}{##1}}}
\expandafter\def\csname PY@tok@ow\endcsname{\let\PY@bf=\textbf\def\PY@tc##1{\textcolor[rgb]{0.67,0.13,1.00}{##1}}}
\expandafter\def\csname PY@tok@nb\endcsname{\def\PY@tc##1{\textcolor[rgb]{0.00,0.50,0.00}{##1}}}
\expandafter\def\csname PY@tok@nf\endcsname{\def\PY@tc##1{\textcolor[rgb]{0.00,0.00,1.00}{##1}}}
\expandafter\def\csname PY@tok@nc\endcsname{\let\PY@bf=\textbf\def\PY@tc##1{\textcolor[rgb]{0.00,0.00,1.00}{##1}}}
\expandafter\def\csname PY@tok@nn\endcsname{\let\PY@bf=\textbf\def\PY@tc##1{\textcolor[rgb]{0.00,0.00,1.00}{##1}}}
\expandafter\def\csname PY@tok@ne\endcsname{\let\PY@bf=\textbf\def\PY@tc##1{\textcolor[rgb]{0.82,0.25,0.23}{##1}}}
\expandafter\def\csname PY@tok@nv\endcsname{\def\PY@tc##1{\textcolor[rgb]{0.10,0.09,0.49}{##1}}}
\expandafter\def\csname PY@tok@no\endcsname{\def\PY@tc##1{\textcolor[rgb]{0.53,0.00,0.00}{##1}}}
\expandafter\def\csname PY@tok@nl\endcsname{\def\PY@tc##1{\textcolor[rgb]{0.63,0.63,0.00}{##1}}}
\expandafter\def\csname PY@tok@ni\endcsname{\let\PY@bf=\textbf\def\PY@tc##1{\textcolor[rgb]{0.60,0.60,0.60}{##1}}}
\expandafter\def\csname PY@tok@na\endcsname{\def\PY@tc##1{\textcolor[rgb]{0.49,0.56,0.16}{##1}}}
\expandafter\def\csname PY@tok@nt\endcsname{\let\PY@bf=\textbf\def\PY@tc##1{\textcolor[rgb]{0.00,0.50,0.00}{##1}}}
\expandafter\def\csname PY@tok@nd\endcsname{\def\PY@tc##1{\textcolor[rgb]{0.67,0.13,1.00}{##1}}}
\expandafter\def\csname PY@tok@s\endcsname{\def\PY@tc##1{\textcolor[rgb]{0.73,0.13,0.13}{##1}}}
\expandafter\def\csname PY@tok@sd\endcsname{\let\PY@it=\textit\def\PY@tc##1{\textcolor[rgb]{0.73,0.13,0.13}{##1}}}
\expandafter\def\csname PY@tok@si\endcsname{\let\PY@bf=\textbf\def\PY@tc##1{\textcolor[rgb]{0.73,0.40,0.53}{##1}}}
\expandafter\def\csname PY@tok@se\endcsname{\let\PY@bf=\textbf\def\PY@tc##1{\textcolor[rgb]{0.73,0.40,0.13}{##1}}}
\expandafter\def\csname PY@tok@sr\endcsname{\def\PY@tc##1{\textcolor[rgb]{0.73,0.40,0.53}{##1}}}
\expandafter\def\csname PY@tok@ss\endcsname{\def\PY@tc##1{\textcolor[rgb]{0.10,0.09,0.49}{##1}}}
\expandafter\def\csname PY@tok@sx\endcsname{\def\PY@tc##1{\textcolor[rgb]{0.00,0.50,0.00}{##1}}}
\expandafter\def\csname PY@tok@m\endcsname{\def\PY@tc##1{\textcolor[rgb]{0.40,0.40,0.40}{##1}}}
\expandafter\def\csname PY@tok@gh\endcsname{\let\PY@bf=\textbf\def\PY@tc##1{\textcolor[rgb]{0.00,0.00,0.50}{##1}}}
\expandafter\def\csname PY@tok@gu\endcsname{\let\PY@bf=\textbf\def\PY@tc##1{\textcolor[rgb]{0.50,0.00,0.50}{##1}}}
\expandafter\def\csname PY@tok@gd\endcsname{\def\PY@tc##1{\textcolor[rgb]{0.63,0.00,0.00}{##1}}}
\expandafter\def\csname PY@tok@gi\endcsname{\def\PY@tc##1{\textcolor[rgb]{0.00,0.63,0.00}{##1}}}
\expandafter\def\csname PY@tok@gr\endcsname{\def\PY@tc##1{\textcolor[rgb]{1.00,0.00,0.00}{##1}}}
\expandafter\def\csname PY@tok@ge\endcsname{\let\PY@it=\textit}
\expandafter\def\csname PY@tok@gs\endcsname{\let\PY@bf=\textbf}
\expandafter\def\csname PY@tok@gp\endcsname{\let\PY@bf=\textbf\def\PY@tc##1{\textcolor[rgb]{0.00,0.00,0.50}{##1}}}
\expandafter\def\csname PY@tok@go\endcsname{\def\PY@tc##1{\textcolor[rgb]{0.53,0.53,0.53}{##1}}}
\expandafter\def\csname PY@tok@gt\endcsname{\def\PY@tc##1{\textcolor[rgb]{0.00,0.27,0.87}{##1}}}
\expandafter\def\csname PY@tok@err\endcsname{\def\PY@bc##1{\setlength{\fboxsep}{0pt}\fcolorbox[rgb]{1.00,0.00,0.00}{1,1,1}{\strut ##1}}}
\expandafter\def\csname PY@tok@kc\endcsname{\let\PY@bf=\textbf\def\PY@tc##1{\textcolor[rgb]{0.00,0.50,0.00}{##1}}}
\expandafter\def\csname PY@tok@kd\endcsname{\let\PY@bf=\textbf\def\PY@tc##1{\textcolor[rgb]{0.00,0.50,0.00}{##1}}}
\expandafter\def\csname PY@tok@kn\endcsname{\let\PY@bf=\textbf\def\PY@tc##1{\textcolor[rgb]{0.00,0.50,0.00}{##1}}}
\expandafter\def\csname PY@tok@kr\endcsname{\let\PY@bf=\textbf\def\PY@tc##1{\textcolor[rgb]{0.00,0.50,0.00}{##1}}}
\expandafter\def\csname PY@tok@bp\endcsname{\def\PY@tc##1{\textcolor[rgb]{0.00,0.50,0.00}{##1}}}
\expandafter\def\csname PY@tok@fm\endcsname{\def\PY@tc##1{\textcolor[rgb]{0.00,0.00,1.00}{##1}}}
\expandafter\def\csname PY@tok@vc\endcsname{\def\PY@tc##1{\textcolor[rgb]{0.10,0.09,0.49}{##1}}}
\expandafter\def\csname PY@tok@vg\endcsname{\def\PY@tc##1{\textcolor[rgb]{0.10,0.09,0.49}{##1}}}
\expandafter\def\csname PY@tok@vi\endcsname{\def\PY@tc##1{\textcolor[rgb]{0.10,0.09,0.49}{##1}}}
\expandafter\def\csname PY@tok@vm\endcsname{\def\PY@tc##1{\textcolor[rgb]{0.10,0.09,0.49}{##1}}}
\expandafter\def\csname PY@tok@sa\endcsname{\def\PY@tc##1{\textcolor[rgb]{0.73,0.13,0.13}{##1}}}
\expandafter\def\csname PY@tok@sb\endcsname{\def\PY@tc##1{\textcolor[rgb]{0.73,0.13,0.13}{##1}}}
\expandafter\def\csname PY@tok@sc\endcsname{\def\PY@tc##1{\textcolor[rgb]{0.73,0.13,0.13}{##1}}}
\expandafter\def\csname PY@tok@dl\endcsname{\def\PY@tc##1{\textcolor[rgb]{0.73,0.13,0.13}{##1}}}
\expandafter\def\csname PY@tok@s2\endcsname{\def\PY@tc##1{\textcolor[rgb]{0.73,0.13,0.13}{##1}}}
\expandafter\def\csname PY@tok@sh\endcsname{\def\PY@tc##1{\textcolor[rgb]{0.73,0.13,0.13}{##1}}}
\expandafter\def\csname PY@tok@s1\endcsname{\def\PY@tc##1{\textcolor[rgb]{0.73,0.13,0.13}{##1}}}
\expandafter\def\csname PY@tok@mb\endcsname{\def\PY@tc##1{\textcolor[rgb]{0.40,0.40,0.40}{##1}}}
\expandafter\def\csname PY@tok@mf\endcsname{\def\PY@tc##1{\textcolor[rgb]{0.40,0.40,0.40}{##1}}}
\expandafter\def\csname PY@tok@mh\endcsname{\def\PY@tc##1{\textcolor[rgb]{0.40,0.40,0.40}{##1}}}
\expandafter\def\csname PY@tok@mi\endcsname{\def\PY@tc##1{\textcolor[rgb]{0.40,0.40,0.40}{##1}}}
\expandafter\def\csname PY@tok@il\endcsname{\def\PY@tc##1{\textcolor[rgb]{0.40,0.40,0.40}{##1}}}
\expandafter\def\csname PY@tok@mo\endcsname{\def\PY@tc##1{\textcolor[rgb]{0.40,0.40,0.40}{##1}}}
\expandafter\def\csname PY@tok@ch\endcsname{\let\PY@it=\textit\def\PY@tc##1{\textcolor[rgb]{0.25,0.50,0.50}{##1}}}
\expandafter\def\csname PY@tok@cm\endcsname{\let\PY@it=\textit\def\PY@tc##1{\textcolor[rgb]{0.25,0.50,0.50}{##1}}}
\expandafter\def\csname PY@tok@cpf\endcsname{\let\PY@it=\textit\def\PY@tc##1{\textcolor[rgb]{0.25,0.50,0.50}{##1}}}
\expandafter\def\csname PY@tok@c1\endcsname{\let\PY@it=\textit\def\PY@tc##1{\textcolor[rgb]{0.25,0.50,0.50}{##1}}}
\expandafter\def\csname PY@tok@cs\endcsname{\let\PY@it=\textit\def\PY@tc##1{\textcolor[rgb]{0.25,0.50,0.50}{##1}}}

\def\PYZbs{\char`\\}
\def\PYZus{\char`\_}
\def\PYZob{\char`\{}
\def\PYZcb{\char`\}}
\def\PYZca{\char`\^}
\def\PYZam{\char`\&}
\def\PYZlt{\char`\<}
\def\PYZgt{\char`\>}
\def\PYZsh{\char`\#}
\def\PYZpc{\char`\%}
\def\PYZdl{\char`\$}
\def\PYZhy{\char`\-}
\def\PYZsq{\char`\'}
\def\PYZdq{\char`\"}
\def\PYZti{\char`\~}
% for compatibility with earlier versions
\def\PYZat{@}
\def\PYZlb{[}
\def\PYZrb{]}
\makeatother


    % Exact colors from NB
    \definecolor{incolor}{rgb}{0.0, 0.0, 0.5}
    \definecolor{outcolor}{rgb}{0.545, 0.0, 0.0}



    
    % Prevent overflowing lines due to hard-to-break entities
    \sloppy 
    % Setup hyperref package
    \hypersetup{
      breaklinks=true,  % so long urls are correctly broken across lines
      colorlinks=true,
      urlcolor=urlcolor,
      linkcolor=linkcolor,
      citecolor=citecolor,
      }
    % Slightly bigger margins than the latex defaults
    
    \geometry{verbose,tmargin=1in,bmargin=1in,lmargin=1in,rmargin=1in}
    
    

    \begin{document}
    
    
    \maketitle
    
    

    
    \section{DSI Technical Exercise, Question
3}\label{dsi-technical-exercise-question-3}

    \subsection{Business Case Study for Experiment and
Analysis}\label{business-case-study-for-experiment-and-analysis}

\begin{itemize}
\tightlist
\item
  A consumer posts a \textbf{request} for a service needed. Every
  request is in some \textbf{category} (e.g., Catering, Personal
  Training, Interior Design) and some \textbf{location} (e.g., New York,
  San Francisco).\\
\item
  We match the request up with appropriate service providers and send
  each of those providers an \textbf{invite} to quote on the request.\\
\item
  Providers view the invite and some choose to send a \textbf{quote} to
  the consumer expressing interest.
\end{itemize}

    \subsection{Split Test (or A/B/n) Experimental Design and Test
Results}\label{split-test-or-abn-experimental-design-and-test-results}

I've just concluded a test of our \emph{quote form}. After receiving an
invite, service providers come to the quote form to view the consumer
request and choose whether or not to pay to send a quote. My goal was to
determine if certain changes to the design of the form would cause more
providers to send a quote after coming to the page.

Over the course of a week, I divided invites from about 3000 requests
among four new variations of the quote form as well as the baseline form
we've been using for the last year. Here are my results:

\begin{itemize}
\tightlist
\item
  Baseline: 32 quotes out of 595 viewers
\item
  Variation 1: 30 quotes out of 599 viewers
\item
  Variation 2: 18 quotes out of 622 viewers
\item
  Variation 3: 51 quotes out of 606 viewers
\item
  Variation 4: 38 quotes out of 578 viewers
\end{itemize}

    \subsection{Split Test (or A/B/n)
Analysis}\label{split-test-or-abn-analysis}

    \paragraph{Analytical Assumptions}\label{analytical-assumptions}

As population inferences will be made using the sample statistics
generated as part of the test execution these assumptions must be made:
the sample statistic distribution is approximately normal, the samples
are independant, and the sample size is significantly large.

The significance level, or \(\alpha\), will be 5\% for this analysis.

    \paragraph{Experimental Design}\label{experimental-design}

The summary of the experimental design overview indicates human
involvement with the assignment of service providers and a particular
summary page. Further discussion is needed to ensure that the assignment
process is sufficiently randomize.

    \begin{Verbatim}[commandchars=\\\{\}]
{\color{incolor}In [{\color{incolor}34}]:} \PY{c+c1}{\PYZsh{} import standard libraries for analytical work}
         \PY{k+kn}{import} \PY{n+nn}{pandas} \PY{k}{as} \PY{n+nn}{pd}
         \PY{k+kn}{import} \PY{n+nn}{statsmodels}\PY{n+nn}{.}\PY{n+nn}{api} \PY{k}{as} \PY{n+nn}{sms}
\end{Verbatim}


    \begin{Verbatim}[commandchars=\\\{\}]
{\color{incolor}In [{\color{incolor}35}]:} \PY{c+c1}{\PYZsh{} load csv test results into a dataframe}
         \PY{n}{df} \PY{o}{=} \PY{n}{pd}\PY{o}{.}\PY{n}{read\PYZus{}csv}\PY{p}{(}\PY{l+s+s2}{\PYZdq{}}\PY{l+s+s2}{./test\PYZus{}data/acme\PYZus{}corp.csv}\PY{l+s+s2}{\PYZdq{}}\PY{p}{)}
         \PY{c+c1}{\PYZsh{} take a look at the dataframe}
         \PY{n}{df}\PY{o}{.}\PY{n}{head}\PY{p}{(}\PY{p}{)}
\end{Verbatim}


\begin{Verbatim}[commandchars=\\\{\}]
{\color{outcolor}Out[{\color{outcolor}35}]:}         Bucket  Quotes  Views
         0     Baseline      32    595
         1  Variation 1      30    599
         2  Variation 2      18    622
         3  Variation 3      51    606
         4  Variation 4      38    578
\end{Verbatim}
            
    \paragraph{The Hypotheses}\label{the-hypotheses}

    The A/B/n (or Split Test) testing analysis is a controlled experiment
leveraging hypothesis statements. Let's go ahead and document all four
null/alternative statements.

Null and alternative hypothesis for the controlled experiment of the
variation 1 of the submittal page versus the baseline or existing
submittal page.

\(H_{0}: \pi_{variation1} - \pi_{baseline} = 0\)\\
\(H_{a}: \pi_{variation1} - \pi_{baseline} \gt 0\)

Null and alternative hypothesis for the controlled experiment of the
variation 2 of the submittal page versus the baseline or existing
submittal page.

\(H_{0}: \pi_{variation2} - \pi_{baseline} = 0\)\\
\(H_{a}: \pi_{variation2} - \pi_{baseline} \gt 0\)

Null and alternative hypothesis for the controlled experiment of the
variation 3 of the submittal page versus the baseline or existing
submittal page.

\(H_{0}: \pi_{variation3} - \pi_{baseline} = 0\)\\
\(H_{a}: \pi_{variation3} - \pi_{baseline} \gt 0\)

Null and alternative hypothesis for the controlled experiment of the
variation 4 of the submittal page versus the baseline or existing
submittal page.

\(H_{0}: \pi_{variation4} - \pi_{baseline} = 0\)\\
\(H_{a}: \pi_{variation4} - \pi_{baseline} \gt 0\)

Each of these null statements assert that no difference exists between
the population proportion, where proportion is represented as \(\pi\).

All 4 of the alternative statements assert that the population
proportion is greater than the existing population proportion, where
proportion is represented as \(\pi\).

    \paragraph{Built-In Function
"proportions\_ztest"}\label{built-in-function-proportions_ztest}

    The A/B/n testing analysis will utilize a built-in statsmodel function
called proportions\_ztest. Looking at the help page for this function
several arguments are required for this calculation, which will be
performed in the next cell.

    \begin{Verbatim}[commandchars=\\\{\}]
{\color{incolor}In [{\color{incolor}36}]:} \PY{c+c1}{\PYZsh{} baseline count and nobs}
         \PY{n}{baseline\PYZus{}df} \PY{o}{=} \PY{n}{df}\PY{o}{.}\PY{n}{query}\PY{p}{(}\PY{l+s+s2}{\PYZdq{}}\PY{l+s+s2}{Bucket == }\PY{l+s+s2}{\PYZsq{}}\PY{l+s+s2}{Baseline}\PY{l+s+s2}{\PYZsq{}}\PY{l+s+s2}{\PYZdq{}}\PY{p}{)}
         \PY{n}{baseline\PYZus{}count} \PY{o}{=} \PY{n}{baseline\PYZus{}df}\PY{o}{.}\PY{n}{Quotes}\PY{o}{.}\PY{n}{sum}\PY{p}{(}\PY{p}{)}
         \PY{n}{baseline\PYZus{}nobs} \PY{o}{=} \PY{n}{baseline\PYZus{}df}\PY{o}{.}\PY{n}{Views}\PY{o}{.}\PY{n}{sum}\PY{p}{(}\PY{p}{)}
         \PY{n+nb}{print}\PY{p}{(}\PY{n}{f}\PY{l+s+s2}{\PYZdq{}}\PY{l+s+s2}{Baseline count and nobs is: }\PY{l+s+si}{\PYZob{}baseline\PYZus{}count\PYZcb{}}\PY{l+s+s2}{ and }\PY{l+s+si}{\PYZob{}baseline\PYZus{}nobs\PYZcb{}}\PY{l+s+s2}{\PYZdq{}}\PY{p}{)}
         
         \PY{c+c1}{\PYZsh{} variation 1 count and nobs}
         \PY{n}{variation1\PYZus{}df} \PY{o}{=} \PY{n}{df}\PY{o}{.}\PY{n}{query}\PY{p}{(}\PY{l+s+s2}{\PYZdq{}}\PY{l+s+s2}{Bucket == }\PY{l+s+s2}{\PYZsq{}}\PY{l+s+s2}{Variation 1}\PY{l+s+s2}{\PYZsq{}}\PY{l+s+s2}{\PYZdq{}}\PY{p}{)}
         \PY{n}{variation1\PYZus{}count} \PY{o}{=} \PY{n}{variation1\PYZus{}df}\PY{o}{.}\PY{n}{Quotes}\PY{o}{.}\PY{n}{sum}\PY{p}{(}\PY{p}{)}
         \PY{n}{variation1\PYZus{}nobs} \PY{o}{=} \PY{n}{variation1\PYZus{}df}\PY{o}{.}\PY{n}{Views}\PY{o}{.}\PY{n}{sum}\PY{p}{(}\PY{p}{)}
         \PY{n+nb}{print}\PY{p}{(}\PY{n}{f}\PY{l+s+s2}{\PYZdq{}}\PY{l+s+s2}{Variation 1 count and nobs is: }\PY{l+s+si}{\PYZob{}variation1\PYZus{}count\PYZcb{}}\PY{l+s+s2}{ and }\PY{l+s+si}{\PYZob{}variation1\PYZus{}nobs\PYZcb{}}\PY{l+s+s2}{\PYZdq{}}\PY{p}{)}
         
         \PY{c+c1}{\PYZsh{} variation 2 count and nobs}
         \PY{n}{variation2\PYZus{}df} \PY{o}{=} \PY{n}{df}\PY{o}{.}\PY{n}{query}\PY{p}{(}\PY{l+s+s2}{\PYZdq{}}\PY{l+s+s2}{Bucket == }\PY{l+s+s2}{\PYZsq{}}\PY{l+s+s2}{Variation 2}\PY{l+s+s2}{\PYZsq{}}\PY{l+s+s2}{\PYZdq{}}\PY{p}{)}
         \PY{n}{variation2\PYZus{}count} \PY{o}{=} \PY{n}{variation2\PYZus{}df}\PY{o}{.}\PY{n}{Quotes}\PY{o}{.}\PY{n}{sum}\PY{p}{(}\PY{p}{)}
         \PY{n}{variation2\PYZus{}nobs} \PY{o}{=} \PY{n}{variation2\PYZus{}df}\PY{o}{.}\PY{n}{Views}\PY{o}{.}\PY{n}{sum}\PY{p}{(}\PY{p}{)}
         \PY{n+nb}{print}\PY{p}{(}\PY{n}{f}\PY{l+s+s2}{\PYZdq{}}\PY{l+s+s2}{Variation 2 count and nobs is: }\PY{l+s+si}{\PYZob{}variation2\PYZus{}count\PYZcb{}}\PY{l+s+s2}{ and }\PY{l+s+si}{\PYZob{}variation2\PYZus{}nobs\PYZcb{}}\PY{l+s+s2}{\PYZdq{}}\PY{p}{)}
         
         \PY{c+c1}{\PYZsh{} variation 3 count and nobs}
         \PY{n}{variation3\PYZus{}df} \PY{o}{=} \PY{n}{df}\PY{o}{.}\PY{n}{query}\PY{p}{(}\PY{l+s+s2}{\PYZdq{}}\PY{l+s+s2}{Bucket == }\PY{l+s+s2}{\PYZsq{}}\PY{l+s+s2}{Variation 3}\PY{l+s+s2}{\PYZsq{}}\PY{l+s+s2}{\PYZdq{}}\PY{p}{)}
         \PY{n}{variation3\PYZus{}count} \PY{o}{=} \PY{n}{variation3\PYZus{}df}\PY{o}{.}\PY{n}{Quotes}\PY{o}{.}\PY{n}{sum}\PY{p}{(}\PY{p}{)}
         \PY{n}{variation3\PYZus{}nobs} \PY{o}{=} \PY{n}{variation3\PYZus{}df}\PY{o}{.}\PY{n}{Views}\PY{o}{.}\PY{n}{sum}\PY{p}{(}\PY{p}{)}
         \PY{n+nb}{print}\PY{p}{(}\PY{n}{f}\PY{l+s+s2}{\PYZdq{}}\PY{l+s+s2}{Variation 3 count and nobs is: }\PY{l+s+si}{\PYZob{}variation3\PYZus{}count\PYZcb{}}\PY{l+s+s2}{ and }\PY{l+s+si}{\PYZob{}variation3\PYZus{}nobs\PYZcb{}}\PY{l+s+s2}{\PYZdq{}}\PY{p}{)}
         
         \PY{c+c1}{\PYZsh{} variation 3 count and nobs}
         \PY{n}{variation4\PYZus{}df} \PY{o}{=} \PY{n}{df}\PY{o}{.}\PY{n}{query}\PY{p}{(}\PY{l+s+s2}{\PYZdq{}}\PY{l+s+s2}{Bucket == }\PY{l+s+s2}{\PYZsq{}}\PY{l+s+s2}{Variation 4}\PY{l+s+s2}{\PYZsq{}}\PY{l+s+s2}{\PYZdq{}}\PY{p}{)}
         \PY{n}{variation4\PYZus{}count} \PY{o}{=} \PY{n}{variation4\PYZus{}df}\PY{o}{.}\PY{n}{Quotes}\PY{o}{.}\PY{n}{sum}\PY{p}{(}\PY{p}{)}
         \PY{n}{variation4\PYZus{}nobs} \PY{o}{=} \PY{n}{variation4\PYZus{}df}\PY{o}{.}\PY{n}{Views}\PY{o}{.}\PY{n}{sum}\PY{p}{(}\PY{p}{)}
         \PY{n+nb}{print}\PY{p}{(}\PY{n}{f}\PY{l+s+s2}{\PYZdq{}}\PY{l+s+s2}{Variation 4 count and nobs is: }\PY{l+s+si}{\PYZob{}variation4\PYZus{}count\PYZcb{}}\PY{l+s+s2}{ and }\PY{l+s+si}{\PYZob{}variation4\PYZus{}nobs\PYZcb{}}\PY{l+s+s2}{\PYZdq{}}\PY{p}{)}
\end{Verbatim}


    \begin{Verbatim}[commandchars=\\\{\}]
Baseline count and nobs is: 32 and 595
Variation 1 count and nobs is: 30 and 599
Variation 2 count and nobs is: 18 and 622
Variation 3 count and nobs is: 51 and 606
Variation 4 count and nobs is: 38 and 578

    \end{Verbatim}

    \begin{Verbatim}[commandchars=\\\{\}]
{\color{incolor}In [{\color{incolor}37}]:} \PY{c+c1}{\PYZsh{} calculate the z\PYZhy{}score and p\PYZhy{}value for variation1 and baseline experiment}
         \PY{n}{z\PYZus{}score\PYZus{}variation1}\PY{p}{,} \PY{n}{p\PYZus{}value\PYZus{}variation1} \PY{o}{=} \PY{n}{sms}\PY{o}{.}\PY{n}{stats}\PY{o}{.}\PY{n}{proportions\PYZus{}ztest}\PY{p}{(}\PY{p}{[}\PY{n}{variation1\PYZus{}count}\PY{p}{,} \PY{n}{baseline\PYZus{}count}\PY{p}{]}\PY{p}{,} \PY{p}{[}\PY{n}{variation1\PYZus{}nobs}\PY{p}{,} \PY{n}{baseline\PYZus{}nobs}\PY{p}{]}\PY{p}{,} \PY{n}{alternative}\PY{o}{=}\PY{l+s+s2}{\PYZdq{}}\PY{l+s+s2}{larger}\PY{l+s+s2}{\PYZdq{}}\PY{p}{)}
         \PY{n+nb}{print}\PY{p}{(}\PY{n}{f}\PY{l+s+s2}{\PYZdq{}}\PY{l+s+s2}{The p\PYZhy{}value for the Variation 1 and Baseline experiment is }\PY{l+s+si}{\PYZob{}p\PYZus{}value\PYZus{}variation1\PYZcb{}}\PY{l+s+s2}{\PYZdq{}}\PY{p}{)}
         
         \PY{c+c1}{\PYZsh{} calculate the z\PYZhy{}score and p\PYZhy{}value for variation2 and baseline experiment}
         \PY{n}{z\PYZus{}score\PYZus{}variation2}\PY{p}{,} \PY{n}{p\PYZus{}value\PYZus{}variation2} \PY{o}{=} \PY{n}{sms}\PY{o}{.}\PY{n}{stats}\PY{o}{.}\PY{n}{proportions\PYZus{}ztest}\PY{p}{(}\PY{p}{[}\PY{n}{variation2\PYZus{}count}\PY{p}{,} \PY{n}{baseline\PYZus{}count}\PY{p}{]}\PY{p}{,} \PY{p}{[}\PY{n}{variation2\PYZus{}nobs}\PY{p}{,} \PY{n}{baseline\PYZus{}nobs}\PY{p}{]}\PY{p}{,} \PY{n}{alternative}\PY{o}{=}\PY{l+s+s2}{\PYZdq{}}\PY{l+s+s2}{larger}\PY{l+s+s2}{\PYZdq{}}\PY{p}{)}
         \PY{n+nb}{print}\PY{p}{(}\PY{n}{f}\PY{l+s+s2}{\PYZdq{}}\PY{l+s+s2}{The p\PYZhy{}value for the Variation 2 and Baseline experiment is }\PY{l+s+si}{\PYZob{}p\PYZus{}value\PYZus{}variation2\PYZcb{}}\PY{l+s+s2}{\PYZdq{}}\PY{p}{)}
         
         \PY{c+c1}{\PYZsh{} calculate the z\PYZhy{}score and p\PYZhy{}value for variation3 and baseline experiment}
         \PY{n}{z\PYZus{}score\PYZus{}variation3}\PY{p}{,} \PY{n}{p\PYZus{}value\PYZus{}variation3} \PY{o}{=} \PY{n}{sms}\PY{o}{.}\PY{n}{stats}\PY{o}{.}\PY{n}{proportions\PYZus{}ztest}\PY{p}{(}\PY{p}{[}\PY{n}{variation3\PYZus{}count}\PY{p}{,} \PY{n}{baseline\PYZus{}count}\PY{p}{]}\PY{p}{,} \PY{p}{[}\PY{n}{variation3\PYZus{}nobs}\PY{p}{,} \PY{n}{baseline\PYZus{}nobs}\PY{p}{]}\PY{p}{,} \PY{n}{alternative}\PY{o}{=}\PY{l+s+s2}{\PYZdq{}}\PY{l+s+s2}{larger}\PY{l+s+s2}{\PYZdq{}}\PY{p}{)}
         \PY{n+nb}{print}\PY{p}{(}\PY{n}{f}\PY{l+s+s2}{\PYZdq{}}\PY{l+s+s2}{The p\PYZhy{}value for the Variation 3 and Baseline experiment is }\PY{l+s+si}{\PYZob{}p\PYZus{}value\PYZus{}variation3\PYZcb{}}\PY{l+s+s2}{\PYZdq{}}\PY{p}{)}
         
         \PY{c+c1}{\PYZsh{} calculate the z\PYZhy{}score and p\PYZhy{}value for variation4 and baseline experiment}
         \PY{n}{z\PYZus{}score\PYZus{}variation4}\PY{p}{,} \PY{n}{p\PYZus{}value\PYZus{}variation4} \PY{o}{=} \PY{n}{sms}\PY{o}{.}\PY{n}{stats}\PY{o}{.}\PY{n}{proportions\PYZus{}ztest}\PY{p}{(}\PY{p}{[}\PY{n}{variation4\PYZus{}count}\PY{p}{,} \PY{n}{baseline\PYZus{}count}\PY{p}{]}\PY{p}{,} \PY{p}{[}\PY{n}{variation4\PYZus{}nobs}\PY{p}{,} \PY{n}{baseline\PYZus{}nobs}\PY{p}{]}\PY{p}{,} \PY{n}{alternative}\PY{o}{=}\PY{l+s+s2}{\PYZdq{}}\PY{l+s+s2}{larger}\PY{l+s+s2}{\PYZdq{}}\PY{p}{)}
         \PY{n+nb}{print}\PY{p}{(}\PY{n}{f}\PY{l+s+s2}{\PYZdq{}}\PY{l+s+s2}{The p\PYZhy{}value for the Variation 4 and Baseline experiment is }\PY{l+s+si}{\PYZob{}p\PYZus{}value\PYZus{}variation4\PYZcb{}}\PY{l+s+s2}{\PYZdq{}}\PY{p}{)}
\end{Verbatim}


    \begin{Verbatim}[commandchars=\\\{\}]
The p-value for the Variation 1 and Baseline experiment is 0.6133099143530945
The p-value for the Variation 2 and Baseline experiment is 0.9854676438929962
The p-value for the Variation 3 and Baseline experiment is 0.018986239169411456
The p-value for the Variation 4 and Baseline experiment is 0.19360793468873894

    \end{Verbatim}

    \paragraph{Conclusions}\label{conclusions}

The p-value for Variation 1, 2, and 4 experiment is greater than the
significance level of 5\% and the null hyposthesis will not be rejected.
The p-value for experiment is less than the significance level of 5\%
for the third variation of the submittal page, make that experiment the
only variation allowing us to reject the null hypothesis.

Recommending that Variation 3 of the submittal page be scheduled for
installation into the production domain for customer use.

    \paragraph{References}\label{references}

    \begin{enumerate}
\def\labelenumi{\arabic{enumi}.}
\tightlist
\item
  "A Refresher on A/B Testing". Retrieved from
  https://hbr.org/2017/06/a-refresher-on-ab-testing\#
\item
  "proportions\_ztest". Retrieved from
  http://knowledgetack.com/python/statsmodels/proportions\_ztest/
\end{enumerate}


    % Add a bibliography block to the postdoc
    
    
    
    \end{document}
